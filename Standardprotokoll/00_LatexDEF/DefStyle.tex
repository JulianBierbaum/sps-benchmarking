% 							Style-File - Vorlage für Elektrotechnik
%                         				2023 HTL Weiz
%----------------------------------------------------------------------------------------------------
% Dieses \LaTeX Template kann als Grundlage für die Erstellung von Diplomarbeitsdokumentationen
% verwendet werden. In diesem File wird das Aussehen, sowie Designeinstellungen implementiert.
%----------------------------------------------------------------------------------------------------

%****************************************************************************************************
%==================================== S E I T E N F O R M A T =======================================
%****************************************************************************************************

\usepackage[a4paper, left=2.5cm, right=2.5cm, top=2.5cm, bottom=2.5cm]{geometry}						% A4 und Seitenränder
\setlength{\textwidth}{16.0cm}																			% Textbreite
\setlength{\parindent}{0pt}																				% Einzug ausschalten
\setstretch{1.3}																						% Zeilenabstand
\pagestyle{fancy}																						% Seitenformatierung

%****************************************************************************************************
%=============================== K O P F-  U N D  F U ß Z E I L E ===================================
%****************************************************************************************************

%======================================== S T A N D A R D ===========================================

% Kopfzeile
\lhead{\LARGE Systemtechnisches Laboratorium}						% Baselineskip entfernen falls der Übungsname in eine Zeile passt und das Logo kleiner machen
\rhead{\vspace{-8pt}\includegraphics[scale=0.05]{./00_Introduction/HTL-Weiz-Logo.pdf}}
\renewcommand{\headrulewidth}{0.4pt}

% Fußzeile
\lfoot{SYT}
\cfoot{\iYear}
\rfoot{Seite \thepage\ von \pageref{LastPage}}
\renewcommand{\footrulewidth}{0.4pt}

%****************************************************************************************************
%==================================== F A R B E N F O R M A T =======================================
%****************************************************************************************************

\definecolor{green}{rgb}{0.25, 0.5, 0.35}
\definecolor{orange}{rgb}{0.9,0.64,0.25}
\definecolor{kaki}{rgb}{0.74,0.74,0.088}
\definecolor{blue}{rgb}{0.2,0.59,0.94}
\definecolor{dred}{rgb}{0.722,0.18,0.18}
\definecolor{darkgrey}{rgb}{0.364,0.411,0.439}
\definecolor{lightgrey}{rgb}{0.776,0.792,0.803}

%****************************************************************************************************
%=============================== G L I E D E R U N G S E B E N E N ==================================
%****************************************************************************************************

% \setcounter{secnumdepth}{5}																			% Gliederungstiefe (hier: 5. Unterüberschrift [1.1.1.1.1]) jedoch eher für "BOOK" geeignet
\setcounter{tocdepth}{2}																				% Inhaltsverzeichnistiefe (hier 2. Unterüberschrift [1.1])

%****************************************************************************************************
%============================== B I L D U N T E R S C H R I F T E N =================================
%****************************************************************************************************

\captionsetup{margin=10pt,font=small, labelfont={color=black!80, bf}, textfont={color=black!60}, format=hang, indention=-1cm}
\captionsetup[wrapfigure]{name=Abbildung}
\captionsetup[figure]{name=Abbildung}

%****************************************************************************************************
%================================= T E X T F E L D V O R L A G E ====================================
%****************************************************************************************************

\newenvironment{Textfeld}{\begin{tcolorbox}[colback=lightgrey!5,colframe=darkgrey!100]}{\end{tcolorbox}}

%****************************************************************************************************
%================================== C O D E D A R S T E L U N G =====================================
%****************************************************************************************************

%============================== S T R U K T U R I E R T E R  T E X T ================================

\lstdefinelanguage{ST}
{
	morekeywords={VAR, END_VAR, IF, THEN, ELSE, END_IF, WRITE},
	sensitive=false,
	morecomment=[l]{//},
	morecomment=[s]{/*}{*/},
	morestring=[b]",
}

\lstset{
	language=ST,
	basicstyle=\small\ttfamily,
	keywordstyle=\color{blue},
	commentstyle=\color{green},
	numbers=left,
	numberstyle=\tiny,
	numbersep=5pt,
	frame=single,
	breaklines=true,
	breakatwhitespace=true,
	tabsize=4,
	columns=flexible,
}


\definecolor{arduinoGreen}    {rgb} {0.17, 0.43, 0.01}
\definecolor{arduinoGrey}     {rgb} {0.47, 0.47, 0.33}
\definecolor{arduinoOrange}   {rgb} {0.8 , 0.4 , 0   }
\definecolor{arduinoBlue}     {rgb} {0.01, 0.61, 0.98}
\definecolor{arduinoDarkBlue} {rgb} {0.0 , 0.2 , 0.5 }

%%% Define Arduino Language %%%
\lstdefinelanguage{Arduino}{
	language=C++, % begin with default C++ settings 
	%
	%
	%%% Keyword Color Group 1 %%%  (called KEYWORD3 by arduino)
	keywordstyle=\color{arduinoGreen},   
	deletekeywords={  % remove all arduino keywords that might be in c++
		break, case, override, final, continue, default, do, else, for, 
		if, return, goto, switch, throw, try, while, setup, loop, export, 
		not, or, and, xor, include, define, elif, else, error, if, ifdef, 
		ifndef, pragma, warning,
		HIGH, LOW, INPUT, INPUT_PULLUP, OUTPUT, DEC, BIN, HEX, OCT, PI, 
		HALF_PI, TWO_PI, LSBFIRST, MSBFIRST, CHANGE, FALLING, RISING, 
		DEFAULT, EXTERNAL, INTERNAL, INTERNAL1V1, INTERNAL2V56, LED_BUILTIN, 
		LED_BUILTIN_RX, LED_BUILTIN_TX, DIGITAL_MESSAGE, FIRMATA_STRING, 
		ANALOG_MESSAGE, REPORT_DIGITAL, REPORT_ANALOG, SET_PIN_MODE, 
		SYSTEM_RESET, SYSEX_START, auto, int8_t, int16_t, int32_t, int64_t, 
		uint8_t, uint16_t, uint32_t, uint64_t, char16_t, char32_t, operator, 
		enum, delete, bool, boolean, byte, char, const, false, float, double, 
		null, NULL, int, long, new, private, protected, public, short, 
		signed, static, volatile, String, void, true, unsigned, word, array, 
		sizeof, dynamic_cast, typedef, const_cast, struct, static_cast, union, 
		friend, extern, class, reinterpret_cast, register, explicit, inline, 
		_Bool, complex, _Complex, _Imaginary, atomic_bool, atomic_char, 
		atomic_schar, atomic_uchar, atomic_short, atomic_ushort, atomic_int, 
		atomic_uint, atomic_long, atomic_ulong, atomic_llong, atomic_ullong, 
		virtual, PROGMEM,
		Serial, Serial1, Serial2, Serial3, SerialUSB, Keyboard, Mouse,
		abs, acos, asin, atan, atan2, ceil, constrain, cos, degrees, exp, 
		floor, log, map, max, min, radians, random, randomSeed, round, sin, 
		sq, sqrt, tan, pow, bitRead, bitWrite, bitSet, bitClear, bit, 
		highByte, lowByte, analogReference, analogRead, 
		analogReadResolution, analogWrite, analogWriteResolution, 
		attachInterrupt, detachInterrupt, digitalPinToInterrupt, delay, 
		delayMicroseconds, digitalWrite, digitalRead, interrupts, millis, 
		micros, noInterrupts, noTone, pinMode, pulseIn, pulseInLong, shiftIn, 
		shiftOut, tone, yield, Stream, begin, end, peek, read, print, 
		println, available, availableForWrite, flush, setTimeout, find, 
		findUntil, parseInt, parseFloat, readBytes, readBytesUntil, readString, 
		readStringUntil, trim, toUpperCase, toLowerCase, charAt, compareTo, 
		concat, endsWith, startsWith, equals, equalsIgnoreCase, getBytes, 
		indexOf, lastIndexOf, length, replace, setCharAt, substring, 
		toCharArray, toInt, press, release, releaseAll, accept, click, move, 
		isPressed, isAlphaNumeric, isAlpha, isAscii, isWhitespace, isControl, 
		isDigit, isGraph, isLowerCase, isPrintable, isPunct, isSpace, 
		isUpperCase, isHexadecimalDigit, 
	}, 
	morekeywords={   % add arduino structures to group 1
		break, case, override, final, continue, default, do, else, for, 
		if, return, goto, switch, throw, try, while, setup, loop, export, 
		not, or, and, xor, include, define, elif, else, error, if, ifdef, 
		ifndef, pragma, warning,
	}, 
	% 
	%
	%%% Keyword Color Group 2 %%%  (called LITERAL1 by arduino)
	keywordstyle=[2]\color{arduinoBlue},   
	keywords=[2]{   % add variables and dataTypes as 2nd group  
		HIGH, LOW, INPUT, INPUT_PULLUP, OUTPUT, DEC, BIN, HEX, OCT, PI, 
		HALF_PI, TWO_PI, LSBFIRST, MSBFIRST, CHANGE, FALLING, RISING, 
		DEFAULT, EXTERNAL, INTERNAL, INTERNAL1V1, INTERNAL2V56, LED_BUILTIN, 
		LED_BUILTIN_RX, LED_BUILTIN_TX, DIGITAL_MESSAGE, FIRMATA_STRING, 
		ANALOG_MESSAGE, REPORT_DIGITAL, REPORT_ANALOG, SET_PIN_MODE, 
		SYSTEM_RESET, SYSEX_START, auto, int8_t, int16_t, int32_t, int64_t, 
		uint8_t, uint16_t, uint32_t, uint64_t, char16_t, char32_t, operator, 
		enum, delete, bool, boolean, byte, char, const, false, float, double, 
		null, NULL, int, long, new, private, protected, public, short, 
		signed, static, volatile, String, void, true, unsigned, word, array, 
		sizeof, dynamic_cast, typedef, const_cast, struct, static_cast, union, 
		friend, extern, class, reinterpret_cast, register, explicit, inline, 
		_Bool, complex, _Complex, _Imaginary, atomic_bool, atomic_char, 
		atomic_schar, atomic_uchar, atomic_short, atomic_ushort, atomic_int, 
		atomic_uint, atomic_long, atomic_ulong, atomic_llong, atomic_ullong, 
		virtual, PROGMEM,
	},  
	% 
	%
	%%% Keyword Color Group 3 %%%  (called KEYWORD1 by arduino)
	keywordstyle=[3]\bfseries\color{arduinoOrange},
	keywords=[3]{  % add built-in functions as a 3rd group
		Serial, Serial1, Serial2, Serial3, SerialUSB, Keyboard, Mouse,
	},      
	%
	%
	%%% Keyword Color Group 4 %%%  (called KEYWORD2 by arduino)
	keywordstyle=[4]\color{arduinoOrange},
	keywords=[4]{  % add more built-in functions as a 4th group
		abs, acos, asin, atan, atan2, ceil, constrain, cos, degrees, exp, 
		floor, log, map, max, min, radians, random, randomSeed, round, sin, 
		sq, sqrt, tan, pow, bitRead, bitWrite, bitSet, bitClear, bit, 
		highByte, lowByte, analogReference, analogRead, 
		analogReadResolution, analogWrite, analogWriteResolution, 
		attachInterrupt, detachInterrupt, digitalPinToInterrupt, delay, 
		delayMicroseconds, digitalWrite, digitalRead, interrupts, millis, 
		micros, noInterrupts, noTone, pinMode, pulseIn, pulseInLong, shiftIn, 
		shiftOut, tone, yield, Stream, begin, end, peek, read, print, 
		println, available, availableForWrite, flush, setTimeout, find, 
		findUntil, parseInt, parseFloat, readBytes, readBytesUntil, readString, 
		readStringUntil, trim, toUpperCase, toLowerCase, charAt, compareTo, 
		concat, endsWith, startsWith, equals, equalsIgnoreCase, getBytes, 
		indexOf, lastIndexOf, length, replace, setCharAt, substring, 
		toCharArray, toInt, press, release, releaseAll, accept, click, move, 
		isPressed, isAlphaNumeric, isAlpha, isAscii, isWhitespace, isControl, 
		isDigit, isGraph, isLowerCase, isPrintable, isPunct, isSpace, 
		isUpperCase, isHexadecimalDigit, 
	},      
	%
	%
	%%% Set Other Colors %%%
	stringstyle=\color{arduinoDarkBlue},    
	commentstyle=\color{arduinoGrey},    
	%          
	%   
	%%%% Line Numbering %%%%
	numbers=left,                    
	numbersep=5pt,                   
	numberstyle=\color{arduinoGrey},    
	%stepnumber=2,                      % show every 2 line numbers
	%
	%
	%%%% Code Box Style %%%%
	breaklines=true,                    % wordwrapping
	tabsize=2,         
	basicstyle=\ttfamily  
}

