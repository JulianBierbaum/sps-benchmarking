% 								Einleitung - Vorlage für Elektrotechnik
%                         					2023 HTL Weiz
%----------------------------------------------------------------------------------------------------
% Dieses \LaTeX Template kann als Grundlage für die Erstellung von Diplomarbeitsdokumentationen
% verwendet werden. In diesem File wird eine einheitliche Einleitung abgehandelt.
%----------------------------------------------------------------------------------------------------

%****************************************************************************************************
%====================================== T I T E L S E I T E =========================================
%****************************************************************************************************

%=========================== K O P F Z E I L E - T I T E L S E I T E ================================

\begin{titlingpage}
	%Kopfzeile
	\begin{figure}[h]
		\begin{minipage}{0.2\textwidth}
			\raggedright
			\includegraphics[width=\linewidth]{./00_Introduction/HTL-Weiz-Logo.pdf}						% HTL-Weiz-Logo
		\end{minipage}
		\hfill
		\begin{minipage}{0.1\textwidth}
			\raggedleft
			\includegraphics[width=\linewidth]{./00_Introduction/HTL-Weiz-IT-Logo.png}					% HTL-Weiz-ET-Logo
		\end{minipage}
		\hfill
		\begin{center}
			\vspace{-1.5cm}
			\centering
			\large \textbf{Höhere Technische\\ Bundeslehranstalt Weiz}									% HTBLA Weiz
		\end{center}
	\end{figure}
	
	\vspace{1cm}
	
	%========================== D O K U M E N T I N F O R M A T I O N E N ==============================
	
	\begin{center} {\resizebox{12cm}{1.2cm}{\textbf{Laboratorium}}}\\ \vspace{0.25cm} {\large Höhere Abteilung für Informationstechnologie}\\ \end{center}	% Diplomarbeit
	\noindent\rule{\textwidth}{0.4pt}
	\vspace{0.3cm}
	\begin{center} {\LARGE\textbf{\iLUName}}\\ \end{center}																							% Diplomarbeitsname
	\vspace{0.3cm}
	\noindent\rule{\textwidth}{0.4pt}
	
	\renewcommand{\arraystretch}{1.4}
	\begin{table}[H]
		\begin{tabular}{p{5.5cm}p{3.28cm}p{3.5cm}p{2cm}}
			\textbf{Übungsdatum}      		& \textbf{Klasse} 									& \textbf{Jahrgang}							& \textbf{Gruppe}		\\
			\iDate                    		& \iClass            								& \iYear									& \iGroup				\\
											&                 									&											&						\\
			\textbf{Schüler}				&													&											&						\\
			\multicolumn{4}{l}{\iNameOne}																															\\
			\multicolumn{4}{l}{\iNameTwo}																															\\
			\multicolumn{4}{l}{\iNameThree}																															\\
%			\multicolumn{4}{l}{\iNameFour}																															\\
\\
											&													&											&						\\
			\textbf{Lehrperson}       		&						               				&            	      	 					&						\\
			\multicolumn{4}{l}{\iCoach}		                    																									\\
											&                 									&                   						&						\\ \hline
			\textbf{Abgabevermerk}  		&                 									&                  							&						\\
											&                 									&                   						&						\\
			Unterschrift:                   & Datum:   											& \hspace{1.5cm} Benotung:					&						\\ \hline
		\end{tabular}
	\end{table}
	\renewcommand{\arraystretch}{1}
\end{titlingpage}
