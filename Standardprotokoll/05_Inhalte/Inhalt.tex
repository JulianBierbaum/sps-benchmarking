% 									Inhalte - Vorlage für Elektrotechnik
%                        					 2023 HTL Weiz
%----------------------------------------------------------------------------------------------------
% Dieses \LaTeX Template kann als Grundlage für die Erstellung von Laborberichtdokumentationen
% verwendet werden. In diesem File werden die Inhalte in das Dokument eingebettet.
%----------------------------------------------------------------------------------------------------

%\begin{figure}[H]
%	\centering
%	\includegraphics[width=5cm]{./01_Inhalte/grafik}
%	\caption{Beschreibung}
%\end{figure}

%================================= A U F G A B E N S T E L L U N G ==================================

\section{Aufgabenstellung}
Ziel ist die Evaluation der Performance von drei Kommunikationsprotokollen im Zusammenspiel mit einer Siemens-S7-SPS: OPC UA, die Siemens Web API sowie das proprietäre S7-Protokoll. Der Untersuchungsschwerpunkt liegt auf dem hochfrequenten Schreiben einzelner Variablen (Datentypen: Bool, Int16, Int32) sowie der Übertragung eines Datenblocks von ca. 1~kB. Ergänzend wird ein Pflichtenheft erstellt.

\section{Entwicklungsumgebung}
Die Implementierung erfolgt in Python unter Verwendung der folgenden Bibliotheken:
\begin{itemize}
	\item \texttt{cryptography} ($\ge$ 46.0.3)
	\item \texttt{matplotlib} ($\ge$ 3.10.7)
	\item \texttt{opcua} ($\ge$ 0.98.13)
	\item \texttt{python-dotenv} ($\ge$ 1.2.1)
	\item \texttt{python-snap7} ($\ge$ 2.0.2)
	\item \texttt{requests} ($\ge$ 2.32.5)
	\item \texttt{urllib3} ($\ge$ 2.5.0)
\end{itemize}

\section{Softwarearchitektur}
Zur Anbindung der Protokolle wurde ein Adapter-Pattern realisiert. Alle Protokoll-Adapter implementieren eine gemeinsame abstrakte Basisklasse, um eine einheitliche Schnittstelle zu gewährleisten:

\begin{lstlisting}[language=Python]
	from abc import ABC, abstractmethod
	from typing import Any, Dict, List, Tuple
	
	class ProtocolAdapter(ABC):
	"""Abstract base class for SPS protocol adapters."""
	
	@abstractmethod
	def connect(self) -> None:
	"""Establish connection or login if required."""
	pass
	
	@abstractmethod
	def disconnect(self) -> None:
	"""Close connection or logout if required."""
	pass
	
	@abstractmethod
	def write(self, var: str, value: Any) -> Tuple[Dict, float]:
	"""Write a single value, return response and latency in ms."""
	pass
	
	@abstractmethod
	def read(self, var: str) -> Tuple[Dict, float]:
	"""Read a single value, return response and latency in ms."""
	pass
	
	@abstractmethod
	def write_bulk_data(self, array_data: List[Any]) -> Tuple[Dict, float]:
	"""Write an entire array of bulk data."""
	pass
\end{lstlisting}

\section{Datenübertragungssicherheit}
Da alle untersuchten Protokolle auf dem TCP-Transportprotokoll basieren, wird bei Erhalt einer positiven Rückmeldung (Success Response) von einer vollständigen und korrekten Übertragung ausgegangen. Eine explizite Verifizierung, ob die Werte physisch im Zielspeicher der SPS verbucht wurden, erfolgt im Rahmen dieses Tests nicht.
%======================================= A U S W E R T U N G ========================================

\section{Auswertung}
Hier wird für einen Messpunkt der komplette Rechengang dargestellt. Zuerst die allgemeine Formel anführen, dann konkrete Zahlenwerte einsetzen.







